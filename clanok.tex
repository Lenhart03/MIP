% Metódy inžinierskej práce

\documentclass[10pt,twoside,slovak,a4paper,hidelinks]{article}

\usepackage[slovak]{babel}
%\usepackage[T1]{fontenc}
\usepackage[IL2]{fontenc} % lepšia sadzba písmena Ľ než v T1
\usepackage[utf8]{inputenc}
\usepackage{graphicx}
\usepackage{url} % príkaz \url na formátovanie URL
\usepackage{hyperref} % odkazy v texte budú aktívne (pri niektorých triedach dokumentov spôsobuje posun textu)

\usepackage{cite}
%\usepackage{times}

\pagestyle{headings}

\title{3D Game Engine a jeho význam\thanks{Semestrálny projekt v predmete Metódy inžinierskej práce, ak. rok 2022/23, vedenie: Zuzana Špitálová}} % meno a priezvisko vyučujúceho na cvičeniach

\author{Jan Lenhart\\[2pt]
	{\small Slovenská technická univerzita v Bratislave}\\
	{\small Fakulta informatiky a informačných technológií}\\
	{\small \texttt{xlenhart@stuba.sk}}
}

\date{\small 27. september 2022} % upravte



\begin{document}

\maketitle

\begin{abstract}
\ldots
\end{abstract}



\section{Úvod}
\section{Systemová architektúra}\cite{Czarnecki:Staged}
\subsection{Low level}
\subsubsection{Scene Manager}
\subsubsection{GUI System}
\subsubsection{Physics Engine}
\subsubsection{Animation Engine}
\subsubsection{Object Manager}
\subsubsection{Special Effects}
\subsection{High level}
\subsubsection{Math Library}
\subsubsection{Data Structures}
\subsubsection{Input System}
\subsubsection{Time System}
\subsubsection{Graphics Engine}
\subsubsection{Audio System}
\section{Záver}



%\acknowledgement{Ak niekomu chcete poďakovať\ldots}


% týmto sa generuje zoznam literatúry z obsahu súboru literatura.bib podľa toho, na čo sa v článku odkazujete
\bibliography{literatura}
\bibliographystyle{plain} % prípadne alpha, abbrv alebo hociktorý iný
\end{document}
