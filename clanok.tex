% Metódy inžinierskej práce

\documentclass[10pt,twoside,slovak,a4paper,hidelinks]{article}

\usepackage[slovak]{babel}
%\usepackage[T1]{fontenc}
\usepackage[IL2]{fontenc} % lepšia sadzba písmena Ľ než v T1
\usepackage[utf8]{inputenc}
\usepackage{graphicx}
\usepackage{url} % príkaz \url na formátovanie URL
\usepackage{hyperref} % odkazy v texte budú aktívne (pri niektorých triedach dokumentov spôsobuje posun textu)

\usepackage{cite}
%\usepackage{times}

\pagestyle{headings}

\title{3D Game Engine-y a ich význam\thanks{Semestrálny projekt v predmete Metódy inžinierskej práce, ak. rok 2022/23, vedenie: Zuzana Špitálová}} % meno a priezvisko vyučujúceho na cvičeniach

\author{Jan Lenhart\\[2pt]
	{\small Slovenská technická univerzita v Bratislave}\\
	{\small Fakulta informatiky a informačných technológií}\\
	{\small \texttt{xlenhart@stuba.sk}}
}

\date{\small 8. október 2022} % upravte

\begin{document}

\maketitle

\begin{abstract}
Každodenným vývojom technológie sa posúvajú aj technologické hranice video hier. V októbri 1958 fyzik William Higinbotham vytvoril, čo sa považuje ako prvá video hra. Bolo to niečo podobné hre PONG \cite{FirstGame}. Neskôr v roku 1980 bola vyvinutá prvá 3D hra. Bola to tanková hra, ktorá využívala vektorovú grafiku na 3D realizáciu \cite{First3DGame}. Obe tieto hry mali veľký význam pre gemifikáciu a ako sa blížime k dnešnej dobe, tvorba hier začína byť o dosť zložitejšia, preto na jej uľahčenie vznikli Game Engine-y. Game Engine-y nám umožňujú vyvíjať hry ako také, bez toho, aby sme museli míňať čas na vývoj základných prostriedkov na ktorých je založená každá video hra. Týmto vieme, že aj Game Engine-y majú mimoriadne veľký význam pre gemifikáciu. V tomto článku sa dozviete o kľúčových častiach Game Engine-ov, nástrojov na tvorbu zábavných, ale aj edukačných hier \cite{IEEE}.
\end{abstract}

%V bibliografickom súbore je niekoľko zdrojov z IEEE, ktoré som ešte necitoval, ale mám v pláne z ních čerpať. Ten, ktorý ma primárne usmernil k téme je https://ieeexplore.ieee.org/document/6321770.

\section{Úvod}
\section{Systemová architektúra}
\subsection{Low level}
\subsubsection{Scene Manager}
\subsubsection{GUI System}
\subsubsection{Physics Engine}
\subsubsection{Animation Engine}
\subsubsection{Object Manager}
\subsubsection{Special Effects}
\subsection{High level}
\subsubsection{Math Library}
\subsubsection{Data Structures}
\subsubsection{Input System}
\subsubsection{Time System}
\subsubsection{Graphics Engine}
\subsubsection{Audio System}
\section{Záver}



%\acknowledgement{Ak niekomu chcete poďakovať\ldots}


% týmto sa generuje zoznam literatúry z obsahu súboru literatura.bib podľa toho, na čo sa v článku odkazujete
\bibliography{literatura}
\bibliographystyle{plain} % prípadne alpha, abbrv alebo hociktorý iný
\end{document}
